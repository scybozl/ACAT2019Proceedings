\documentclass[a4paper]{jpconf}
\usepackage{graphicx}

\begin{document}

\title{Trilinear Higgs boson coupling variations for di-Higgs production with full NLO QCD predictions in \texttt{Powheg}}

\author{G.~Heinrich$^1$, S.~Jones$^2$, M.~Kerner$^3$, G.~Luisoni$^1$ and L.~Scyboz$^1$}

\address{$^1$ Max-Planck-Institut f\"ur Physik (Werner-Heisenberg-Institut), F\"ohringer Ring 6, 80805 M\"unchen, Germany}
\address{$^2$ Theoretical Physics Department, CERN, Geneva, Switzerland}
\address{$^3$ Physik-Institut, Universit\"at Z\"urich, Winterthurerstrasse 190, 8057 Z\"urich, Switzerland}
\ead{gudrun@mpp.mpg.de, s.jones@cern.ch, mkerner@physik.uzh.ch, luisonig@gmail.com, scyboz@mpp.mpg.de}


\begin{abstract}
The Higgs couplings to other particles are increasingly well-measured by the ATLAS and CMS experiments. Yet there is still room for improvement in the measurement of the Higgs trilinear self-coupling $\lambda$, mainly due to the low cross-section for Higgs boson pair production. We present inclusive and differential results for the NLO QCD corrections to Higgs pair production with the full top-quark mass dependence, where the Higgs trilinear self-coupling is varied to non-SM values. The calculation of the two-loop virtual contributions has been performed numerically using CPUs and GPUs. The fixed-order calculation is supplemented by parton showering within the \texttt{Powheg-BOX-V2} event generator, and both \texttt{Pythia8} and \texttt{Herwig7} parton-shower algorithms are implemented in a preliminary study of shower effects.
\end{abstract}


\section{Introduction}

\section{Description of the calculation}

\section{Total and differential cross-sections for variations of the trilinear coupling}

\section{Conclusion}

\section*{References}

\bibliography{acat}{}
\bibliographystyle{plain}

\end{document}


