\documentclass[a4paper]{jpconf}
\usepackage{graphicx}
\usepackage{hyperref}
%\usepackage{subcaption}

\newcommand{\GeV}{\ensuremath{\mathrm{\:GeV}}}
\newcommand{\TeV}{\ensuremath{\mathrm{\:TeV}}}
\newcommand{\pythia}{\texttt{Pythia8}}
\newcommand{\herwig}{\texttt{Herwig7}}
\newcommand{\chhh}{\ensuremath{\kappa_{\lambda}}}


\begin{document}

\title{Trilinear Higgs boson coupling variations for di-Higgs production with full NLO QCD predictions in \texttt{Powheg}}

\author{G.~Heinrich$^1$, S.~Jones$^2$, M.~Kerner$^3$, G.~Luisoni$^1$ and L.~Scyboz$^1$}

\address{$^1$ Max-Planck-Institut f\"ur Physik, F\"ohringer Ring 6, 80805 M\"unchen, Germany}
\address{$^2$ Theoretical Physics Department, CERN, Geneva, Switzerland}
\address{$^3$ Physik-Institut, Universit\"at Z\"urich, Winterthurerstrasse 190, 8057 Z\"urich, Switzerland}
\ead{gudrun@mpp.mpg.de, s.jones@cern.ch, mkerner@physik.uzh.ch, luisonig@gmail.com, scyboz@mpp.mpg.de}


\begin{abstract}
The couplings of the Higgs boson to other particles are increasingly well measured by the ATLAS and CMS experiments. The Higgs boson trilinear self-coupling however is still largely unconstrained,  mainly due to the low cross-section for Higgs boson pair production. We present inclusive and differential results for the NLO QCD corrections to Higgs boson pair production with the full top-quark mass dependence, where the Higgs trilinear coupling is varied to non-SM values. The fixed-order calculation is supplemented by parton showering within the \texttt{Powheg-BOX-V2} event generator, and both \pythia~and \herwig~parton-shower algorithms are implemented in a preliminary study of shower effects.
\end{abstract}


\section{Introduction}

Impressive experimental constraints have been set on the Higgs boson couplings to vector bosons and heavy fermions~\cite{Khachatryan:2016vau,Aaboud:2017vzb,Sirunyan:2018koj,Sirunyan:2018sgc}. The Higgs potential, in contrast, leaves more room for New Physics. In particular, the Higgs boson trilinear self-coupling $\lambda$ can be experimentally constrained by exclusion limits on Higgs boson pair production $pp \to hh$~\cite{Sirunyan:2018two,Aad:2019uzh}, where the best limit on $\kappa_{\lambda}=\lambda / \lambda_{\mathrm{SM}}$ is currently given by ATLAS with $-5.0 < \kappa_{\lambda} < 12.0$ at $95\%$ confidence level.
%The best experimental limit is given by ATLAS with $-5.0 < \kappa_{\lambda} < 12.1$ at $95\%$ confidence level~\cite{ATLAS-CONF-2018-043}, from a combination of $hh \to b\bar{b}b\bar{b}, b\bar{b}\tau^+ \tau^-$ and $b\bar{b}\gamma\gamma$ channels.
Higher-order corrections to Higgs pair production were first calculated in the heavy top-quark mass limit (HTL) $m_t \to \infty$, where the top-quark degrees of freedom are integrated out~\cite{Dawson:1998py,deFlorian:2013jea,Grigo:2014jma,deFlorian:2016uhr}. The NLO QCD corrections with the full top-quark mass dependence were only computed more recently~\cite{Borowka:2016ehy,Borowka:2016ypz,Baglio:2018lrj}. The latter are based on numerical evaluations of the two-loop contribution to $gg \to hh$. For non-SM values of the Higgs couplings, results were computed at NLO QCD in the full theory for a class of extensions of the SM in Ref.~\cite{Buchalla:2018yce}.

In the following, an implementation of the full NLO QCD corrections into the \texttt{Powheg-BOX-V2} event generator~\cite{Nason:2004rx,Frixione:2007vw,Alioli:2010xd} is presented. In this framework, the Higgs trilinear self-coupling can be varied, as well as the top-Higgs Yukawa coupling. Total cross sections are computed for $\sqrt{s}=13,14$ and $27 \TeV$ at the (HE-)LHC. Differential results are shown for $\sqrt{s}=14 \TeV$. The fixed-order calculation is then matched to both $\pythia$~\cite{Sjostrand:2014zea} and $\herwig$~\cite{Bellm:2015jjp,Bellm:2017bvx} parton showers. For a more detailed description, the reader is referred to Ref.~\cite{Heinrich:2019bkc}.

\section{Description of the calculation}

The calculation is based on the setup presented in Ref.~\cite{Heinrich:2017kxx} for the case of the SM. The leading-order amplitude has been computed analytically. The real-emission contributions were implemented using an interface~\cite{Luisoni:2013kna} between the \texttt{Powheg-BOX} and \textsc{GoSam}~\cite{Cullen:2011ac,Cullen:2014yla}, 
where the reduction of the one-loop amplitude has been performed with \texttt{Ninja}~\cite{Peraro:2014cba}, using master integrals from 
\texttt{golem95C}~\cite{Binoth:2008uq,Cullen:2011kv}, \texttt{OneLOop}~\cite{vanHameren:2010cp} and \texttt{VBFNLO}~\cite{Arnold:2008rz,Baglio:2014uba}.
The two-loop amplitude for the full virtual contribution was adapted from Refs.~\cite{Borowka:2016ehy,Borowka:2016ypz}, which used an extension of the \textsc{GoSam} package to two loops~\cite{Jones:2016bci}. There, the integral reduction was performed with \textsc{Reduze2}~\cite{vonManteuffel:2012np}, and the integrals were numerically evaluated with \textsc{SecDec3}~\cite{Borowka:2015mxa}. For a faster convergence, the integration was performed within a Quasi-Monte-Carlo implementation using a rank-1 shifted lattice rule~\cite{Borowka:2018goh,Jones:theseproceedings}. The integrals were computed with 16 dual \textsc{NVidia Tesla K20X} GPUs. The top-quark and Higgs masses have been set to $m_t=173 \GeV$ and $m_h = 125 \GeV$. Thus, the integrals depend only on the two Mandelstam invariants $\hat{s}$ and $\hat{t}$.

A grid for the two-loop amplitude was constructed in both variables using $5291$ pre-sampled phase-space points. %The trilinear Higgs self-coupling enters the full amplitude through the contribution of triangle-like diagrams $\mathcal{M}_T$, as the one shown at leading order in Fig.~\ref{}.
We split the amplitude in two contributions: diagrams containing the trilinear Higgs coupling are called \textit{triangle-like}, and those that do not are called \textit{box-like} (see Fig.~\ref{fig:lodiagrams} for two diagrams at NLO QCD). 

%{\it Comment: it would be more instructive to show two-loop diagram examples}

\begin{figure}[htb!]
\begin{minipage}[c]{0.5\textwidth}
\centering
\includegraphics[width=0.6\textwidth]{figures/{hh1-2}.eps}
\end{minipage}
%\quad
\begin{minipage}[c]{0.5\textwidth}
\centering
\includegraphics[width=0.6\textwidth]{figures/{hh2-2}.eps}
\end{minipage}
\begin{minipage}[t]{0.5\textwidth}
\centering$\mathcal{M}_T$
\end{minipage}
%\quad
\begin{minipage}[t]{0.5\textwidth}
\centering$\mathcal{M}_B$
\end{minipage}
\caption{Triangle-like (left) and box-like (right) diagrams contribute to the full amplitude. The former contain the Higgs self-coupling, while the latter do not. \label{fig:lodiagrams}}
\end{figure}


At any order in QCD, the squared matrix-element can thus be written as a second-order polynomial in $\lambda$:

\begin{equation}
M_{\lambda} \equiv |{\cal M}_\lambda|^2=\mathcal{M}_B^* \mathcal{M}_B + \lambda \left( \mathcal{M}_B^* \mathcal{M}_T + \mathcal{M}_T^* \mathcal{M}_B  \right) +  \lambda^2 \mathcal{M}_T^* \mathcal{M}_T \;.
\label{eq:2ndorder}
\end{equation}

The two-loop amplitude for an arbitrary value of $\lambda$ can be reconstructed from the squared matrix-element computed for three different values of $\lambda$. In our case, we chose $\chhh=\lambda_{\mathrm{BSM}}/\lambda_{\mathrm{SM}} \in \lbrace -1, 0, 1 \rbrace$. A new grid is generated at runtime for the user-defined value of $\lambda$, where the amplitude for each pre-sampled phase-space point is calculated as:
	
\begin{equation}
M_{\lambda} =M_0\cdot(1-\lambda^2)+\frac{M_1}{2}\cdot(\lambda+\lambda^2) + \frac{M_{-1}}{2}\cdot(-\lambda+\lambda^2)\;.
\end{equation}

The grid produced for the two-loop amplitude is fed to an interpolation framework, which interfaces the result at \textit{any} phase-space point $M_{\lambda}(\hat{s}, \hat{t})$ to \texttt{Powheg}.

%An interpolation framework takes the produced grid as input and the amplitude at \textit{any} phase-space point $M_{\lambda}(\hat{s}, \hat{t})$ can be interfaced to \texttt{Powheg}.



\section{Total and differential cross-sections for variations of the trilinear coupling}

The results given below are produced using the PDF4LHC15\_nlo\_30\_pdfas sets~\cite{Butterworth:2015oua,CT14,MMHT14,NNPDF} interfaced to \texttt{Powheg} via \texttt{LHAPDF6}~\cite{Buckley:2014ana}, with the corresponding value of $\alpha_s$. The top-quark mass is renormalised in the on-shell scheme and is set to $m_t=173 \GeV$, as in the virtual amplitude. The mass of the Higgs boson is fixed to $m_h=125 \GeV$, and the top-quark and Higgs widths are set to zero. Jets are clustered using the anti-$k_T$ algorithm~\cite{Cacciari:2008gp} as implemented in \texttt{FastJet}~\cite{Cacciari:2005hq, Cacciari:2011ma}, with a jet distance parameter of $R=0.4$ and a minimum transverse momentum requirement of $p_{T} = 20 \GeV$. The central renormalisation and factorisation scales are set to $\mu_R = \mu_F = \mu_0 = m_{\mathrm{hh}} / 2$. Scale uncertainties are estimated by 3-point variations $\mu_R = \mu_F = c\, \mu_0$, with $c \in \lbrace {0.5, 1.0, 2.0} \rbrace$.

Total cross-sections for Higgs pair production at the (HE-)LHC are shown in Table~\ref{tab:sigmatot}, for centre-of-mass energies of $\sqrt{s}=13,14$ and $27 \TeV$ and different values of the Higgs self-coupling $\chhh = \lambda_{\mathrm{BSM}} / \lambda_{\mathrm{SM}}$. They are accompanied by their relative scale uncertainties, which are of the order $\mathcal{O}(10-20\%)$. Notably, the $K$-factors at $14 \TeV$ show a sizeable dependence on the trilinear coupling $\chhh$. In the HTL at NLO QCD, Ref.~\cite{Grober:2015cwa} suggested a variation of the $K$-factors with $\chhh$ of the order $\mathcal{O}(2-3\%)$. In the full theory, the $K$-factors are found to vary between $1.56$ and $2.15$ for variations of the trilinear coupling $-5 \leq \chhh \leq 12$, see Fig.~\ref{fig:Kfacvariation}. 

\begin{table}[htb!]
\begin{center}
%\setlength{\extrarowheight}{3.0pt}
\begin{tabular}{| c | c | c |c|c|}
%\Xhline{2\arrayrulewidth}
\hline
&&&&\\
$\lambda_{\mathrm{BSM}}/\lambda_{\mathrm{SM}}$ & $\sigma_{\rm{NLO}}@13 \mathrm{TeV}$\,[fb]& $\sigma_{\rm{NLO}}@14 \mathrm{TeV}$\,[fb] & $\sigma_{\rm{NLO}}@27 \mathrm{TeV}$\,[fb] &K-factor@14TeV\\
&&&&\\
\hline
-1& 116.71$^{+16.4\%}_{-14.3\%}$  & 136.91$^{+16.4\%}_{-13.9\%}$& 504.9$^{+14.1\%}_{-11.8\%}$ & 1.86 \\
\hline
0& 62.51$^{+15.8\%}_{-13.7\%}$ & 73.64$^{+15.4\%}_{-13.4\%}$& 275.29$^{+13.2\%}_{-11.3\%}$& 1.79  \\
\hline 
1& 27.84$^{+11.6\%}_{-12.9\%}$ & 32.88$^{+13.5\%}_{-12.5\%}$&127.7$^{+11.5\%}_{-10.4\%}$ &1.66\\
\hline
2 & 12.42$^{+13.1\%}_{-12.0\%}$ & 14.75$^{+12.0\%}_{-11.8\%}$ &  59.10$^{+10.2\%}_{-9.7\%}$ & 1.56 \\
\hline
2.4& 11.65$^{+13.9\%}_{-12.7\%}$ & 13.79$^{+13.5\%}_{-12.5\%}$& 53.67$^{+11.4\%}_{-10.3\%}$ & 1.65 \\
\hline
3& 16.28$^{+16.2\%}_{-15.3\%}$ & 19.07$^{+17.1\%}_{-14.1\%}$ & 69.84$^{+14.6\%}_{-12.1\%}$ & 1.90 \\
\hline 
5& 81.74$^{+20.0\%}_{-15.6\%}$  & 95.22$^{+19.7\%}_{-11.5\%}$& 330.61$^{+17.4\%}_{-13.6\%}$ & 2.14 \\
\hline 
\end{tabular}
\end{center}
\caption{Total cross-sections  for Higgs boson pair production at NLO QCD at (HE-)LHC for centre-of-mass energies of $\sqrt{s}=13,14$ and $27 \TeV$. The scale uncertainties are given in percent.
\label{tab:sigmatot}}
\end{table}

\begin{figure}[htb!]
  \centering
    \includegraphics[width=0.65\textwidth]{figures/Kfactor.pdf}
%  \includegraphics[width=\textwidth]{figures/}
%    \caption{\label{fig:lambda_large_27}}
\caption{The dependence of the $K$-factor on the trilinear Higgs self-couplings $\chhh$ is given at $\sqrt{s}=14$\,TeV in the full theory.}
\label{fig:Kfacvariation}
\end{figure}

In Fig.~\ref{fig:lambdavar14TeV}, distributions of the invariant mass $m_{\mathrm{hh}}$ of the Higgs boson pair system are displayed for different values of $\chhh$. They exhibit a characteristic dip around $m_{\mathrm{hh}} \sim 350 \GeV$ for values of the trilinear coupling around $\chhh = 2.4$. This value of the trilinear self-coupling corresponds to a maximally destructive interference between triangle-like and box-like diagrams. For $\chhh = 1$, the maximal destructive interference happens at the $hh$ production threshold and therefore does not manifest itself as a dip, while  for $\chhh$ values larger than $\sim 3$ the triangle-type contributions start to dominate.


\begin{figure}[htb!]
\begin{minipage}[(a)]{0.5\textwidth}
\includegraphics[width=1.\textwidth]{figures/{NLO_cHHH_1_2_2.4_0_mHH-paper}.pdf}
\end{minipage}
\hfill
\begin{minipage}[(a)]{0.5\textwidth}
\includegraphics[width=1.\textwidth]{figures/NLO_cHHH_1_-1_3_5_mHH-paper.pdf}
\end{minipage}
\caption{Distributions of the Higgs boson pair invariant mass $m_{hh}$ for various
  values of $\chhh$ at $\sqrt{s}=14$\,TeV. %Small positive values of $\chhh$ are given on the left, and large or negative values on the right.
  The uncertainty bands are from
  scale variations as described in the text.\label{fig:lambdavar14TeV}}
\end{figure}

Note that since the contributions can be separated in triangle- and box-like diagrams, the top-Higgs Yukawa coupling $y_t$ can easily be varied within the same code. A non-SM value of $y_t$ yields in Eq.~(\ref{eq:2ndorder}):

\begin{equation}
|{\cal M}_\lambda|^2=y_t^4 \left[ \mathcal{M}_B^* \mathcal{M}_B + \frac{\chhh}{y_t} \left( \mathcal{M}_B^* \mathcal{M}_T + \mathcal{M}_T^* \mathcal{M}_B  \right) +  \left( \frac{\chhh}{y_t} \right)^2 \mathcal{M}_T^* \mathcal{M}_T \right] \;.
\end{equation}

The cross-section can be computed by setting $\chhh$ in the code to the desired value of the ratio $\chhh / y_t$, and rescaling the result by an overall factor $y_t^4$. For example, $\sigma(y_t=1.2,\chhh=1)=(1.2)^4\,\sigma(y_t=1,\chhh=1/1.2)$. Fig.~\ref{fig:ytvars}  shows the distribution of $m_{\mathrm{hh}}$ for  values of the top-Higgs Yukawa coupling that are still not experimentally excluded~\cite{Sirunyan:2018sgc}.

\begin{figure}[htb!]
  \centering
    \includegraphics[width=0.65\textwidth]{figures/{NLO_CPP_yt_0.8_1.2_mHH}.pdf}
\caption{The distribution of the Higgs boson pair invariant mass $m_{\mathrm{hh}}$ for values of the top-Higgs Yukawa coupling $y_t \in \lbrace 0.8, 1, 1.2 \rbrace$.}
\label{fig:ytvars}
\end{figure}

\section{Parton-shower matched results}

We now consider NLO distributions matched to a parton shower. The Les Houches Events (LHE)~\cite{Alwall:2006yp} files produced by \texttt{Powheg} are used as input to the \texttt{Pythia8.235} and \texttt{Herwig7.1.4} parton showers. In the case of \herwig, both the default angular-ordered $\tilde{q}$ and the dipole showers are compared. The radiation-regulating \texttt{hdamp} parameter in \texttt{Powheg} is set to $\texttt{hdamp}=250 \GeV$. Multiple-parton interactions and hadronisation are switched off. The default tunes are used for both parton showers.

%We compare the fixed-order NLO result to predictions matched to \pythia~(PP8) and the \herwig~$\tilde{q}-$ordered (PH7-$\tilde{q}$), respectively dipole (PH7-dipole) showers. 

Fig.~\ref{fig:lambdavar14TeV_pTHH_dRHH_showers} displays the transverse momentum of the Higgs boson pair $p_T^{\mathrm{hh}}$ and the separation between the two Higgs bosons $\Delta R^{\mathrm{hh}} = \sqrt{(\eta_1 - \eta_2)^2 + (\phi_1 - \phi_2)^2}$. Considering first the distribution of $p_T^{\mathrm{hh}}$, both \herwig~parton showers (PH7-$\tilde{q}$ and PH7-dipole) generate similar results and reproduce the fixed-order NLO prediction in the far-$p_T^{\mathrm{hh}}$ range. In contrast, \pythia~agrees with \herwig~only for small transverse momenta, while it produces much harder radiation in the tail of the distribution. The same comments apply to the $\Delta R^{\mathrm{hh}}$ observable in the region $0 < \Delta R^{\mathrm{hh}} < \pi$ where shower contributions are important. Large parton-shower matching uncertainties in Higgs boson pair production have already been discussed in Ref.~\cite{Jones:2017giv}. 

\begin{figure}[htb!]
\begin{minipage}[(a)]{0.5\textwidth}
\includegraphics[width=1.\textwidth]{figures/{NLO_PP8_PH7_PH7D_cHHH_1_ptHH}.pdf}
\end{minipage}
\hfill
\begin{minipage}[(a)]{0.5\textwidth}
\includegraphics[width=1.\textwidth]{figures/{NLO_PP8_PH7_PH7D_cHHH_1_dRHH}.pdf}
\end{minipage}
\caption{The transverse momentum $p_T^{\mathrm{hh}}$ of the Higgs boson pair and the separation between the two Higgs bosons $\Delta R^{\mathrm{hh}}$ are shown for the fixed-order NLO calculation and three parton showers, in 
the $\chhh=1$ case.\label{fig:lambdavar14TeV_pTHH_dRHH_showers}}
\end{figure}


\section{Conclusion}

We have presented a new program package for Higgs boson pair production at NLO QCD with full top-quark mass dependence. In this package, the trilinear Higgs self-coupling can be varied explicitly. Within the same code, simultaneous variations of the top-Higgs Yukawa coupling can also be produced. The public code for the \texttt{Powheg-BOX-V2} event generator can be found at the website \href{http://powhegbox.mib.infn.it}{\tt http://powhegbox.mib.infn.it} in the {\tt User-Processes-V2/ggHH} subdirectory. In addition, approximations related to the heavy top limit (HTL) can be enabled for comparison purposes. We have found that the full $m_t$-dependent NLO QCD corrections lead to $K$-factors which exhibit a sizeable dependence  on the value of the trilinear Higgs self-coupling, which is not present in the HTL. We have compared fixed-order predictions at NLO QCD to parton-shower matched results. Both the \pythia~and \herwig~($\tilde{q}$ and dipole) parton showers can be matched directly to LHE files produced by \texttt{Powheg}. Full particle-level events can be produced with our framework, including Higgs boson decays and hadronisation.

\section*{Acknowledgments}

This research was supported in part by the COST Action CA16201 'Particleface' of the European Union, and by the Swiss National Science Foundation (SNF) under grant number 200020-175595. 

\section*{References}

\bibliographystyle{iopart-num}
\bibliography{acat}{}

\end{document}


